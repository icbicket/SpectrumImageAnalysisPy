\documentclass[12pt, titlepage]{article}

\usepackage{amsmath, mathtools}

\usepackage{amsfonts}
\usepackage{amssymb}
\usepackage{graphicx}
\usepackage{colortbl}
\usepackage{xr}
\usepackage{xr-hyper}
\usepackage{hyperref}
\usepackage{longtable}
\usepackage{xfrac}
\usepackage{tabularx}
\usepackage{float}
\usepackage{siunitx}
\usepackage{booktabs}
\usepackage{multirow}
\usepackage[section]{placeins}
\usepackage{caption}
\usepackage{fullpage}


\hypersetup{
bookmarks=true,     % show bookmarks bar?
colorlinks=true,       % false: boxed links; true: colored links
linkcolor=red,          % color of internal links (change box color with linkbordercolor)
citecolor=blue,      % color of links to bibliography
filecolor=magenta,  % color of file links
urlcolor=cyan          % color of external links
}
\externaldocument{../../SRS/SRS}
\usepackage{array}

\input{../../Comments}

\newcommand{\progname}{SpectrumImageAnalysisPy}

\begin{document}
\label{doc:MIS}
\bibliographystyle{ieeetr}
\title{Module Interface Specification for \progname{}}

\author{Isobel Bicket}

\date{\today}

\maketitle

\pagenumbering{roman}

\section{Revision History}

\begin{tabularx}{\textwidth}{p{4cm}p{2cm}X}
\toprule {\bf Date} & {\bf Version} & {\bf Notes}\\
\midrule
	November 29, 2017 & 1.0 & Initial draft\\
\bottomrule
\end{tabularx}

~\newpage

\section{Symbols, Abbreviations and Acronyms}

See \hyperref[doc:SRS]{SRS} documentation at \url{https://github.com/icbicket/SpectrumImageAnalysisPy/blob/SpectrumImageAnalysisPy_dev/Doc/SRS/SRS.pdf}.

\newpage

\tableofcontents

\newpage

\pagenumbering{arabic}

\section{Introduction}

The following document details the Module Interface Specifications for
\progname, a library created for the data processing of spectrum image datasets.

Complementary documents include the System Requirement Specifications
and Module Guide.  The full documentation and implementation can be
found at \url{https://github.com/icbicket/SpectrumImageAnalysisPy/tree/SpectrumImageAnalysisPy_dev}.

\section{Notation}

\wss{You should describe your notation.  You can use what is below as
  a starting point.}

The structure of the MIS for modules comes from \cite{HoffmanAndStrooper1995},
with the addition that template modules have been adapted from
\cite{GhezziEtAl2003}.  The mathematical notation comes from Chapter 3 of
\cite{HoffmanAndStrooper1995}.  For instance, the symbol := is used for a
multiple assignment statement and conditional rules follow the form $(c_1
\Rightarrow r_1 | c_2 \Rightarrow r_2 | ... | c_n \Rightarrow r_n )$.

The following table summarizes the primitive data types used by \progname. 

\begin{center}
\renewcommand{\arraystretch}{1.2}
\noindent 
\begin{tabular}{l l p{7.5cm}} 
\toprule 
\textbf{Data Type} & \textbf{Notation} & \textbf{Description}\\ 
\midrule
character & char & a single symbol or digit\\
integer & $\mathbb{Z}$ & a number without a fractional component in (-$\infty$, $\infty$) \\
natural number & $\mathbb{N}$ & a number without a fractional component in [1, $\infty$) \\
real & $\mathbb{R}$ & any number in (-$\infty$, $\infty$)\\
\bottomrule
\end{tabular} 
\end{center}

\noindent
The specification of \progname \ uses some derived data types: sequences, strings, and
tuples. Sequences are lists filled with elements of the same data type. Strings
are sequences of characters. Tuples contain a list of values, potentially of
different types. In addition, \progname \ uses functions, which
are defined by the data types of their inputs and outputs. Local functions are
described by giving their type signature followed by their specification.

\section{Module Decomposition}

The following table is taken directly from the Module Guide document for this project.

\begin{table}[h!]
\centering
\begin{tabular}{p{0.25\textwidth} p{0.25\textwidth} p{0.4\textwidth}}
\toprule
\textbf{Level 1} & \textbf{Level 2} & \textbf{Level 3}\\
\midrule

{Hardware-Hiding Module} & ~ & ~ \\
\midrule

\multirow{18}{0.25\textwidth}{Behaviour-Hiding Module} &
\multirow{4}{0.25\textwidth}{Import} & csv\\
& & dm3\\
& & h5\\
& & rpl\\\cline{2-3}
& \multirow{4}{0.25\textwidth}{Export} & csv\\
& & h5\\
& & png\\
& & rpl\\\cline{2-3}
& \multirow{4}{0.25\textwidth}{Data processing} & Richardson-Lucy
Deconvolution\\
& & Normalization\\
& & Gain correction\\
& & Background correction\\\cline{2-3}
& \multirow{3}{0.25\textwidth}{Data extraction} & 1D slice\\
& & 2D mask\\
& & 3D mask\\\cline{2-3}
& \multirow{3}{0.25\textwidth}{Display} & 1D spectrum plot\\
& & 2D image plot\\
& & 3D spectrum image plot\\
\midrule

\multirow{2}{0.25\textwidth}{Software Decision Module} &
\multirow{3}{0.25\textwidth}{Data} & Spectrum\\
& & Image\\
& & Spectrum Image\\\cline{2-3}
& Array Data Structure\\\cline{2-3}
& Plotting Library\\

\bottomrule

\end{tabular}
\caption{Module Hierarchy}
\label{TblMH}
\end{table}

\newpage
~\newpage

\section{MIS of Hardware Hiding Module} \label{Mod:HH} \wss{Use labels for cross-referencing}

\subsection{Module}

HardwareHiding

\subsection{Uses}


\subsection{Syntax}

\subsubsection{Exported Access Programs}

\begin{center}
\begin{tabular}{p{2cm} p{4cm} p{4cm} p{2cm}}
\hline
\textbf{Name} & \textbf{In} & \textbf{Out} & \textbf{Exceptions} \\
\hline
\wss{accessProg} & - & - & - \\
\hline
\end{tabular}
\end{center}

\subsection{Semantics}

\subsubsection{State Variables}


\subsubsection{Access Routine Semantics}

\noindent \wss{accessProg}():
\begin{itemize}
\item transition: \wss{if appropriate} 
\item output: \wss{if appropriate} 
\item exception: \wss{if appropriate} 
\end{itemize}

\section{MIS of Behaviour Hiding Module} \label{Mod:BehaviourHiding} \wss{Use labels for cross-referencing}

\subsection{Module}

BehaviourHiding

\subsection{Uses}


\subsection{Syntax}

\subsubsection{Exported Access Programs}

\begin{center}
\begin{tabular}{p{2cm} p{4cm} p{4cm} p{2cm}}
\hline
\textbf{Name} & \textbf{In} & \textbf{Out} & \textbf{Exceptions} \\
\hline
\wss{accessProg} & - & - & - \\
\hline
\end{tabular}
\end{center}

\subsection{Semantics}

\subsubsection{State Variables}


\subsubsection{Access Routine Semantics}

\noindent \wss{accessProg}():
\begin{itemize}
\item transition: \wss{if appropriate} 
\item output: \wss{if appropriate} 
\item exception: \wss{if appropriate} 
\end{itemize}

\section{MIS of Import csv Module} \label{Mod:ImportCSV} 
\subsection{Module}
ImportCSV

\subsection{Uses}
\begin{itemize}
\item \hyperref[Mod:Spectrum]{Data 1D Spectrum}
\item \hyperref[Mod:Array]{Array data structure}
\item \hyperref[Mod:HH]{Hardware-hiding}
\end{itemize}

\subsection{Syntax}

\subsubsection{Exported Access Programs}

\begin{center}
\begin{tabular}{p{2cm} p{4cm} p{4cm} p{5cm}}
\hline
\textbf{Name} & \textbf{In} & \textbf{Out} & \textbf{Exceptions} \\
\hline
ReadCSV & fname: str & Spectrum & NO FILE, NOT CSV\\
\hline
\end{tabular}
\end{center}

\subsection{Semantics}

\subsubsection{State Variables}
N/A

\subsubsection{Environment Variables}
filesystem

\subsubsection{Access Routine Semantics}

\noindent ReadCSV():\\
ReadCSV reads a .csv file and creates a Spectrum object with the appropriate assignations to intensity and energy range.
\begin{itemize}
	\item input: fname: str
	\item transition: N/A 
	\item output: \hyperref[Mod:Spectrum]{Spectrum}
	\item exceptions:
\end{itemize}
\begin{center}
	\begin{tabular}{p{3cm} p{12cm}}
		\toprule[0.15em]
		\textbf{Exception} & \textbf{Condition}\\
		\midrule[0.1em]
		\multirow{2}{0.25\textwidth}{NO FILE} & The filename does not correspond to any file in the filesystem\\ 
		& $fname \notin filesystem$\\ 
		\midrule[0.05em]
		\multirow{2}{0.25\textwidth}{NOT CSV} & The indicated file is not a *.csv format\\
		& $fname \notin \{files|files \in .csv\}$\\ 
		
		\bottomrule[0.15em]
	\end{tabular}
\end{center}


\section{MIS of Import dm3 Module} \label{Mod:ImportDM3}

\subsection{Module}

ImportDM3

\subsection{Uses}
\begin{itemize}
	\item Array data structure
	\item Hardware hiding
	\item Data Spectrum Image
\end{itemize}

\subsection{Syntax}

\subsubsection{Exported Access Programs}

\begin{center}
\begin{tabular}{p{2cm} p{4cm} p{4cm} p{4cm}}
\hline
\textbf{Name} & \textbf{In} & \textbf{Out} & \textbf{Exceptions} \\
\hline
ReadDM3 & filename: string & SI: $\mathbb{R}^{X \times Y \times E}$, metadata: dict & NO FILE, WRONG FILETYPE, NO DATA FOUND  \\
\hline
\end{tabular}
\end{center}

\subsection{Semantics}

\subsubsection{State Variables}

\subsubsection{Environment Variables}
\begin{itemize}
\item filedm3
\end{itemize}
\subsubsection{Access Routine Semantics}

\noindent ImportDM3():
\begin{itemize}
\item input: 
\item transition:
\item output: 
\item exception: 
\end{itemize}

\section{MIS of Import h5 Module} \label{Mod:ImportH5} 

\subsection{Module}

ImportH5

\subsection{Uses}


\subsection{Syntax}

\subsubsection{Exported Access Programs}

\begin{center}
\begin{tabular}{p{2cm} p{4cm} p{4cm} p{2cm}}
\hline
\textbf{Name} & \textbf{In} & \textbf{Out} & \textbf{Exceptions} \\
\hline
\wss{accessProg} & - & - & - \\
\hline
\end{tabular}
\end{center}

\subsection{Semantics}

\subsubsection{State Variables}


\subsubsection{Access Routine Semantics}

\noindent \wss{accessProg}():
\begin{itemize}
\item transition: \wss{if appropriate} 
\item output: \wss{if appropriate} 
\item exception: \wss{if appropriate} 
\end{itemize}

\section{MIS of Import rpl Module} \label{Mod:ImportRPL} \wss{Use labels for cross-referencing}

\subsection{Module}

ImportRPL

\subsection{Uses}


\subsection{Syntax}

\subsubsection{Exported Access Programs}

\begin{center}
\begin{tabular}{p{2cm} p{4cm} p{4cm} p{2cm}}
\hline
\textbf{Name} & \textbf{In} & \textbf{Out} & \textbf{Exceptions} \\
\hline
\wss{accessProg} & - & - & - \\
\hline
\end{tabular}
\end{center}

\subsection{Semantics}

\subsubsection{State Variables}


\subsubsection{Access Routine Semantics}

\noindent \wss{accessProg}():
\begin{itemize}
\item transition: \wss{if appropriate} 
\item output: \wss{if appropriate} 
\item exception: \wss{if appropriate} 
\end{itemize}

\section{MIS of Export csv Module} \label{Mod:ExportCSV} \wss{Use labels for cross-referencing}

\subsection{Module}

ExportCSV

\subsection{Uses}


\subsection{Syntax}

\subsubsection{Exported Access Programs}

\begin{center}
\begin{tabular}{p{2cm} p{4cm} p{4cm} p{2cm}}
\hline
\textbf{Name} & \textbf{In} & \textbf{Out} & \textbf{Exceptions} \\
\hline
WriteCSV & - & - & - \\
\hline
\end{tabular}
\end{center}

\subsection{Semantics}

\subsubsection{State Variables}


\subsubsection{Access Routine Semantics}

\noindent WriteCSV():
\begin{itemize}
\item transition: Writes data to csv file 
\item output: csv file
\item exception:
\end{itemize}

\noindent FormatCSV():
\begin{itemize}
\item transition: Formats data to prepare it to write to csv file 
\item output: formatted data
\item exception:
\end{itemize}

\noindent Verify1D():
\begin{itemize}
\item transition: Verifies that the input data is of the correct format (a 1D spectrum) and has a spectral range and an intensity array of equal length
\item output: formatted data
\item exception:
\end{itemize}

\section{MIS of Export h5 Module} \label{Mod:ExportH5} \wss{Use labels for cross-referencing}

\subsection{Module}

ExportH5

\subsection{Uses}


\subsection{Syntax}

\subsubsection{Exported Access Programs}

\begin{center}
\begin{tabular}{p{2cm} p{4cm} p{4cm} p{2cm}}
\hline
\textbf{Name} & \textbf{In} & \textbf{Out} & \textbf{Exceptions} \\
\hline
\wss{accessProg} & - & - & - \\
\hline
\end{tabular}
\end{center}

\subsection{Semantics}

\subsubsection{State Variables}


\subsubsection{Access Routine Semantics}

\noindent \wss{accessProg}():
\begin{itemize}
\item transition: \wss{if appropriate} 
\item output: \wss{if appropriate} 
\item exception: \wss{if appropriate} 
\end{itemize}

\section{MIS of Export png Module} \label{Mod:ExportPNG} \wss{Use labels for cross-referencing}

\subsection{Module}

ExportPNG

\subsection{Uses}


\subsection{Syntax}

\subsubsection{Exported Access Programs}

\begin{center}
\begin{tabular}{p{2cm} p{4cm} p{4cm} p{2cm}}
\hline
\textbf{Name} & \textbf{In} & \textbf{Out} & \textbf{Exceptions} \\
\hline
\wss{accessProg} & - & - & - \\
\hline
\end{tabular}
\end{center}

\subsection{Semantics}

\subsubsection{State Variables}


\subsubsection{Access Routine Semantics}

\noindent \wss{accessProg}():
\begin{itemize}
\item transition: \wss{if appropriate} 
\item output: \wss{if appropriate} 
\item exception: \wss{if appropriate} 
\end{itemize}

\section{MIS of Export rpl Module} \label{Mod:ExportRPL} \wss{Use labels for cross-referencing}

\subsection{Module}

ExportRPL

\subsection{Uses}


\subsection{Syntax}

\subsubsection{Exported Access Programs}

\begin{center}
\begin{tabular}{p{2cm} p{4cm} p{4cm} p{2cm}}
\hline
\textbf{Name} & \textbf{In} & \textbf{Out} & \textbf{Exceptions} \\
\hline
\wss{accessProg} & - & - & - \\
\hline
\end{tabular}
\end{center}

\subsection{Semantics}

\subsubsection{State Variables}


\subsubsection{Access Routine Semantics}

\noindent \wss{accessProg}():
\begin{itemize}
\item transition: \wss{if appropriate} 
\item output: \wss{if appropriate} 
\item exception: \wss{if appropriate} 
\end{itemize}

\section{MIS of Data Processing Richardson-Lucy Deconvolution Module} \label{Mod: RLDeconvolution}

\subsection{Module}

RLDeconvolution

\subsection{Uses}
Array Data Structure

\subsection{Syntax}

\subsubsection{Exported Access Programs}

\begin{center}
\begin{tabular}{p{4cm} p{4cm} p{4cm} p{2cm}}
\hline
\textbf{Name} & \textbf{In} & \textbf{Out} & \textbf{Exceptions} \\
\hline
RLDeconvolution & S, iterations, S, threads & deconvolved SI & - \\
SIDeconvolution & - & - & - \\
\hline
\end{tabular}
\end{center}

\subsection{Semantics}

\subsubsection{State Variables}
N/A
%\begin{itemize}
%\item iterations
%\item PSF
%\item spectrum
%\item threads
%\end{itemize}

\subsubsection{Access Routine Semantics}

\noindent RLDeconvolution():
\begin{itemize}
\item input: S, S, iterations, threads
\item transition:  
\item output: deconvolved spectrum
\item exception: Divide by zero!
\end{itemize}

\noindent SIDeconvolution():
\begin{itemize}
\item input: SI, iterations, S, threads
\item transition: 
\item output: Deconvolved spectrum image
\item exception: divide by zero
\end{itemize}

\section{MIS of Data Processing Normalization Module} \label{Mod:Normalization} \wss{Use labels for cross-referencing}

\subsection{Module}

Normalization

\subsection{Uses}


\subsection{Syntax}

\subsubsection{Exported Access Programs}

\begin{center}
\begin{tabular}{p{2cm} p{4cm} p{4cm} p{2cm}}
\hline
\textbf{Name} & \textbf{In} & \textbf{Out} & \textbf{Exceptions} \\
\hline
\wss{accessProg} & - & - & - \\
\hline
\end{tabular}
\end{center}

\subsection{Semantics}

\subsubsection{State Variables}


\subsubsection{Access Routine Semantics}

\noindent \wss{accessProg}():
\begin{itemize}
\item transition: \wss{if appropriate} 
\item output: \wss{if appropriate} 
\item exception: \wss{if appropriate} 
\end{itemize}

\section{MIS of Data Processing Gain Correction Module} \label{Mod:GainCorr} \wss{Use labels for cross-referencing}

\subsection{Module}

GainCorr

\subsection{Uses}


\subsection{Syntax}

\subsubsection{Exported Access Programs}

\begin{center}
\begin{tabular}{p{2cm} p{4cm} p{4cm} p{2cm}}
\hline
\textbf{Name} & \textbf{In} & \textbf{Out} & \textbf{Exceptions} \\
\hline
\wss{accessProg} & - & - & - \\
\hline
\end{tabular}
\end{center}

\subsection{Semantics}

\subsubsection{State Variables}


\subsubsection{Access Routine Semantics}

\noindent \wss{accessProg}():
\begin{itemize}
\item transition: \wss{if appropriate} 
\item output: \wss{if appropriate} 
\item exception: \wss{if appropriate} 
\end{itemize}

\section{MIS of Data Processing Background Correction Module} \label{Mod:BackgroundCorr} \wss{Use labels for cross-referencing}

\subsection{Module}

BackgroundCorr

\subsection{Uses}


\subsection{Syntax}

\subsubsection{Exported Access Programs}

\begin{center}
\begin{tabular}{p{2cm} p{4cm} p{4cm} p{2cm}}
\hline
\textbf{Name} & \textbf{In} & \textbf{Out} & \textbf{Exceptions} \\
\hline
\wss{accessProg} & - & - & - \\
\hline
\end{tabular}
\end{center}

\subsection{Semantics}

\subsubsection{State Variables}


\subsubsection{Access Routine Semantics}

\noindent \wss{accessProg}():
\begin{itemize}
\item transition: \wss{if appropriate} 
\item output: \wss{if appropriate} 
\item exception: \wss{if appropriate} 
\end{itemize}

\section{MIS of Data Extraction 1D Slice Module} \label{Mod:Slice1D} \wss{Use labels for cross-referencing}

\subsection{Module}

Slice1D

\subsection{Uses}


\subsection{Syntax}

\subsubsection{Exported Access Programs}

\begin{center}
\begin{tabular}{p{2cm} p{4cm} p{4cm} p{2cm}}
\hline
\textbf{Name} & \textbf{In} & \textbf{Out} & \textbf{Exceptions} \\
\hline
CreateMask & - & - & - \\
ApplyMask & - & - & - \\
\hline
\end{tabular}
\end{center}

\subsection{Semantics}

\subsubsection{State Variables}
\begin{itemize}
\item Mask (2D array of booleans)
\end{itemize}

\subsubsection{Access Routine Semantics}

\noindent CreateMask():
\begin{itemize}
\item transition: Creation of the mask for a 2d dataset - relies on user interaction
\item output:  
\item exception:  
\end{itemize}
\an{should this be here, or in display?}

\noindent ApplyMask():
\begin{itemize}
\item transition: Applies 2d mask to dataset
\item output:  
\item exception:  
\end{itemize}

\section{MIS of Data Extraction 2D Mask Module} \label{Mod:Mask2D} \wss{Use labels for cross-referencing}

\subsection{Module}

Mask2D

\subsection{Uses}


\subsection{Syntax}

\subsubsection{Exported Access Programs}

\begin{center}
\begin{tabular}{p{2cm} p{4cm} p{4cm} p{2cm}}
\hline
\textbf{Name} & \textbf{In} & \textbf{Out} & \textbf{Exceptions} \\
\hline
Create mask & keyboard event, mouse event, data size & 2d bool mask of data size & - \\
Apply mask & & & \\
Modify mask & & & \\

\hline
\end{tabular}
\end{center}

\subsection{Semantics}

\subsubsection{State Variables}
\begin{itemize}
\item mask2D
\end{itemize}

\subsubsection{Access Routine Semantics}

\noindent \wss{accessProg}():
\begin{itemize}
\item transition: \wss{if appropriate} 
\item output: \wss{if appropriate} 
\item exception: \wss{if appropriate} 
\end{itemize}

\section{MIS of Data Extraction 3D Mask Module} \label{Mod:Mask3D} \wss{Use labels for cross-referencing}

\subsection{Module}

Mask3D

\subsection{Uses}


\subsection{Syntax}

\subsubsection{Exported Access Programs}

\begin{center}
\begin{tabular}{p{2cm} p{4cm} p{4cm} p{2cm}}
\hline
\textbf{Name} & \textbf{In} & \textbf{Out} & \textbf{Exceptions} \\
\hline
\wss{accessProg} & - & - & - \\
\hline
\end{tabular}
\end{center}

\subsection{Semantics}

\subsubsection{State Variables}
mask3d

\subsubsection{Access Routine Semantics}

\noindent \wss{accessProg}():
\begin{itemize}
\item transition: \wss{if appropriate} 
\item output: \wss{if appropriate} 
\item exception: \wss{if appropriate} 
\end{itemize}

\section{MIS of Display 1D Spectrum Module} \label{Mod:Disp1D} \wss{Use labels for cross-referencing}

\subsection{Module}

Disp1D

\subsection{Uses}
Data 1D Spectrum
Plotting library

\subsection{Syntax}

\subsubsection{Exported Access Programs}

\begin{center}
\begin{tabular}{p{2cm} p{4cm} p{4cm} p{2cm}}
\hline
\textbf{Name} & \textbf{In} & \textbf{Out} & \textbf{Exceptions} \\
\hline
plot & - & - & - \\
\hline
\end{tabular}
\end{center}

\subsection{Semantics}

\subsubsection{State Variables}

\subsubsection{Environment Variables}
fig

\subsubsection{Access Routine Semantics}

\noindent \wss{accessProg}():
\begin{itemize}
\item transition: \wss{if appropriate} 
\item output: \wss{if appropriate} 
\item exception: \wss{if appropriate} 
\end{itemize}

\section{MIS of Display 2D Image Module} \label{Mod:Disp2D} \wss{Use labels for cross-referencing}

\subsection{Module}

Disp2D

\subsection{Uses}


\subsection{Syntax}

\subsubsection{Exported Access Programs}

\begin{center}
\begin{tabular}{p{2cm} p{4cm} p{4cm} p{2cm}}
\hline
\textbf{Name} & \textbf{In} & \textbf{Out} & \textbf{Exceptions} \\
\hline
\wss{accessProg} & - & - & - \\
\hline
\end{tabular}
\end{center}

\subsection{Semantics}

\subsubsection{State Variables}


\subsubsection{Access Routine Semantics}

\noindent \wss{accessProg}():
\begin{itemize}
\item transition: \wss{if appropriate} 
\item output: \wss{if appropriate} 
\item exception: \wss{if appropriate} 
\end{itemize}

\section{MIS of Display 3D Spectrum Image Module} \label{Mod:Disp3D} \wss{Use labels for cross-referencing}

\subsection{Module}

Disp3D

\subsection{Uses}
\begin{itemize}
	\item Data
	\item Plotting library
	\item 2D image plot
	\item 1D spectrum plot
\end{itemize}

\subsection{Syntax}


\subsubsection{Exported Access Programs}

\begin{center}
\begin{tabular}{p{2cm} p{4cm} p{4cm} p{2cm}}
\hline
\textbf{Name} & \textbf{In} & \textbf{Out} & \textbf{Exceptions} \\
\hline
\wss{accessProg} & - & - & - \\
\hline
\end{tabular}
\end{center}

\subsection{Semantics}

\subsubsection{State Variables}
\begin{itemize}
	\item axis2D image
	\item axis1D spectrum
	\item axis2D mask
	\item axis1D contrast
	\item axis colourbar
	\item polygons
	\item slicer
\end{itemize}
\an{do polygons and slicer belong here, or in the mask2d and slice1d modules?}

\subsubsection{Environment Variables}
\begin{itemize}
\item Plotting window displayed on screen
\item Keyboard keys and mouse buttons
\end{itemize}


\subsubsection{Access Routine Semantics}

\noindent \wss{accessProg}():
\begin{itemize}
\item transition: \wss{if appropriate} 
\item output: \wss{if appropriate} 
\item exception: \wss{if appropriate} 
\end{itemize}

\section{MIS of Data 1D Spectrum Module} \label{Mod:Spectrum} \wss{Use labels for cross-referencing}

\subsection{Module}

Spectrum

\subsection{Uses}


\subsection{Syntax}

\subsubsection{Exported Access Programs}

\begin{center}
\begin{tabular}{p{2cm} p{4cm} p{4cm} p{2cm}}
\hline
\textbf{Name} & \textbf{In} & \textbf{Out} & \textbf{Exceptions} \\
\hline
\wss{accessProg} & - & - & - \\
\hline
\end{tabular}
\end{center}

\subsection{Semantics}

\subsubsection{State Variables}


\subsubsection{Access Routine Semantics}

\noindent \wss{accessProg}():
\begin{itemize}
\item transition: \wss{if appropriate} 
\item output: \wss{if appropriate} 
\item exception: \wss{if appropriate} 
\end{itemize}

%\midrule[0.05em] %
%		\multirow{2}{0.25\textwidth}{WRONG DATA FORMAT} & The data contained within the file is not in the right format for a Spectrum\\
%		& $size(data) \notin \{2 \times N | N \times 2 \}$ \\

\section{MIS of Data 2D Image Module} \label{Mod:Image} \wss{Use labels for cross-referencing}

\subsection{Module}

Image

\subsection{Uses}


\subsection{Syntax}

\subsubsection{Exported Access Programs}

\begin{center}
\begin{tabular}{p{2cm} p{4cm} p{4cm} p{2cm}}
\hline
\textbf{Name} & \textbf{In} & \textbf{Out} & \textbf{Exceptions} \\
\hline
\wss{accessProg} & - & - & - \\
\hline
\end{tabular}
\end{center}

\subsection{Semantics}

\subsubsection{State Variables}


\subsubsection{Access Routine Semantics}

\noindent \wss{accessProg}():
\begin{itemize}
\item transition: 
\item output:
\item exception:
\end{itemize}

\section{MIS of Data 3D Spectrum Image Module} \label{Mod:SI}

\subsection{Template Module}
SI

\subsection{Uses}
\begin{itemize}
	\item Array Data Structure
\end{itemize}


\subsection{Syntax}

\subsubsection{Exported Access Programs}

\begin{center}
\begin{tabular}{p{2cm} p{4cm} p{4cm} p{2cm}}
	\hline
	\textbf{Routine Name} & \textbf{In} & \textbf{Out} & \textbf{Exceptions} \\
	\hline
	 init & data & SI & - \\
	\hline
\end{tabular}
\end{center}

\subsection{Semantics}
\an{Stuff it does, in English}

\subsubsection{State Variables}
\begin{itemize}
	\item data: $\mathbb{R}^{X \times Y \times K}$
	\item Imcal: $\mathbb{R}$
	\item dispersion: $\mathbb{R}$
	\item Srange: $\mathbb{R}^{K}$
	\item ZLP: $\mathbb{Z}$
	\item size: $\mathbb{N}^3$
	\item Slabel: string
	\item Sunit: string
	\item metadata: dict
\end{itemize}

\subsubsection{State Invariant}

\subsubsection{Assumptions}

\subsubsection{Access Routine Semantics}
init
\begin{itemize}
\item input: 
\item transition: Initialize all state variables
\item output: 
\item exception: 
\end{itemize}

\section{MIS of Array Data Structure Module} \label{Mod:Array} \wss{Use labels for cross-referencing}

\subsection{Module}

Array

\subsection{Uses}


\subsection{Syntax}

\subsubsection{Exported Access Programs}

\begin{center}
\begin{tabular}{p{2cm} p{4cm} p{4cm} p{2cm}}
\hline
\textbf{Name} & \textbf{In} & \textbf{Out} & \textbf{Exceptions} \\
\hline
\wss{accessProg} & - & - & - \\
\hline
\end{tabular}
\end{center}

\subsection{Semantics}

\subsubsection{State Variables}


\subsubsection{Access Routine Semantics}

\noindent \wss{accessProg}():
\begin{itemize}
\item transition: \wss{if appropriate} 
\item output: \wss{if appropriate} 
\item exception: \wss{if appropriate} 
\end{itemize}

\section{MIS of Plotting Library Module} \label{Mod:Plotting} \wss{Use labels for cross-referencing}

\subsection{Module}

Plotting

\subsection{Uses}


\subsection{Syntax}

\subsubsection{Exported Access Programs}

\begin{center}
\begin{tabular}{p{2cm} p{4cm} p{4cm} p{2cm}}
\hline
\textbf{Name} & \textbf{In} & \textbf{Out} & \textbf{Exceptions} \\
\hline
\wss{accessProg} & - & - & - \\
\hline
\end{tabular}
\end{center}

\subsection{Semantics}

\subsubsection{State Variables}


\subsubsection{Access Routine Semantics}

\noindent \wss{accessProg}():
\begin{itemize}
\item transition: \wss{if appropriate} 
\item output: \wss{if appropriate} 
\item exception: \wss{if appropriate} 
\end{itemize}

\newpage

\bibliography {MIS}

\newpage

\section{Appendix} \label{Appendix}

\wss{Extra information if required}

\end{document}