\documentclass{article}

\usepackage{tabularx}
\usepackage{booktabs}

\title{CAS 741: Problem Statement\\SpectrumImageAnalysisPy}

\author{Isobel Bicket\\bicketic}

\date{14-09-2017}

\input{../Comments}

\begin{document}

\maketitle

\begin{table}[hp]
\caption{Revision History} \label{TblRevisionHistory}
\begin{tabularx}{\textwidth}{llX}
\toprule
\textbf{Date} & \textbf{Developer(s)} & \textbf{Change}\\
\midrule
14-09-2017 & Isobel Bicket & Initial Draft\\

\bottomrule
\end{tabularx}
\end{table}

In the scientific analysis of a sample, a scientist might be interested in the chemical, structural, or electronic information and in how these properties are distributed throughout the sample. The use of spectro-microscopy techniques allows the two-dimensional spatial mapping of spectroscopy data, giving chemical, structural, or electronic information along the third dimension. The datasets created by these techniques are 3D in nature, with two spatial dimensions and one spectral dimension (eg, energy, wavelength), and is known as a \textit{spectrum image}. Many scientists are used to visualizing one- or two-dimensional datasets, but the navigation and representation of three dimensions on a flat screen or sheet of paper poses additional challenges.

The purpose of the software is to enable the processing, navigation, and visualization of features within a 3D spectrum image. The current options in software designed for handling 3D datasets provide either fast and usable data processing operations or easy visual navigation of the dataset, but do not provide both within one complete package. The current software aims to provide the user with a way to process, navigate, and display their datasets in one simple package. 

Interested stakeholders in this project may include researchers, students, professors, or technicians involved in the processing of three-dimensional spectro-microscopy datasets. The processing encoded into the current software will focus on applications for electron microscopy data, but should be easily expandable to other techniques. The software should run on a variety of personal desktop or laptop computers using Linux, Windows, or MacOS for use by a broad user base.

\wss{Try to set  your editor to an 80 column width, with hard returns at the end
  of each line.  This makes it much easier to see diff's between different
  version of your documentation, and it makes it easier for me to read your
  source code, since my editor does not have auto wrapping.}

\end{document}